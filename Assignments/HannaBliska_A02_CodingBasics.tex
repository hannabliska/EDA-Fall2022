% Options for packages loaded elsewhere
\PassOptionsToPackage{unicode}{hyperref}
\PassOptionsToPackage{hyphens}{url}
%
\documentclass[
]{article}
\usepackage{amsmath,amssymb}
\usepackage{lmodern}
\usepackage{iftex}
\ifPDFTeX
  \usepackage[T1]{fontenc}
  \usepackage[utf8]{inputenc}
  \usepackage{textcomp} % provide euro and other symbols
\else % if luatex or xetex
  \usepackage{unicode-math}
  \defaultfontfeatures{Scale=MatchLowercase}
  \defaultfontfeatures[\rmfamily]{Ligatures=TeX,Scale=1}
\fi
% Use upquote if available, for straight quotes in verbatim environments
\IfFileExists{upquote.sty}{\usepackage{upquote}}{}
\IfFileExists{microtype.sty}{% use microtype if available
  \usepackage[]{microtype}
  \UseMicrotypeSet[protrusion]{basicmath} % disable protrusion for tt fonts
}{}
\makeatletter
\@ifundefined{KOMAClassName}{% if non-KOMA class
  \IfFileExists{parskip.sty}{%
    \usepackage{parskip}
  }{% else
    \setlength{\parindent}{0pt}
    \setlength{\parskip}{6pt plus 2pt minus 1pt}}
}{% if KOMA class
  \KOMAoptions{parskip=half}}
\makeatother
\usepackage{xcolor}
\usepackage[margin=2.54cm]{geometry}
\usepackage{color}
\usepackage{fancyvrb}
\newcommand{\VerbBar}{|}
\newcommand{\VERB}{\Verb[commandchars=\\\{\}]}
\DefineVerbatimEnvironment{Highlighting}{Verbatim}{commandchars=\\\{\}}
% Add ',fontsize=\small' for more characters per line
\usepackage{framed}
\definecolor{shadecolor}{RGB}{248,248,248}
\newenvironment{Shaded}{\begin{snugshade}}{\end{snugshade}}
\newcommand{\AlertTok}[1]{\textcolor[rgb]{0.94,0.16,0.16}{#1}}
\newcommand{\AnnotationTok}[1]{\textcolor[rgb]{0.56,0.35,0.01}{\textbf{\textit{#1}}}}
\newcommand{\AttributeTok}[1]{\textcolor[rgb]{0.77,0.63,0.00}{#1}}
\newcommand{\BaseNTok}[1]{\textcolor[rgb]{0.00,0.00,0.81}{#1}}
\newcommand{\BuiltInTok}[1]{#1}
\newcommand{\CharTok}[1]{\textcolor[rgb]{0.31,0.60,0.02}{#1}}
\newcommand{\CommentTok}[1]{\textcolor[rgb]{0.56,0.35,0.01}{\textit{#1}}}
\newcommand{\CommentVarTok}[1]{\textcolor[rgb]{0.56,0.35,0.01}{\textbf{\textit{#1}}}}
\newcommand{\ConstantTok}[1]{\textcolor[rgb]{0.00,0.00,0.00}{#1}}
\newcommand{\ControlFlowTok}[1]{\textcolor[rgb]{0.13,0.29,0.53}{\textbf{#1}}}
\newcommand{\DataTypeTok}[1]{\textcolor[rgb]{0.13,0.29,0.53}{#1}}
\newcommand{\DecValTok}[1]{\textcolor[rgb]{0.00,0.00,0.81}{#1}}
\newcommand{\DocumentationTok}[1]{\textcolor[rgb]{0.56,0.35,0.01}{\textbf{\textit{#1}}}}
\newcommand{\ErrorTok}[1]{\textcolor[rgb]{0.64,0.00,0.00}{\textbf{#1}}}
\newcommand{\ExtensionTok}[1]{#1}
\newcommand{\FloatTok}[1]{\textcolor[rgb]{0.00,0.00,0.81}{#1}}
\newcommand{\FunctionTok}[1]{\textcolor[rgb]{0.00,0.00,0.00}{#1}}
\newcommand{\ImportTok}[1]{#1}
\newcommand{\InformationTok}[1]{\textcolor[rgb]{0.56,0.35,0.01}{\textbf{\textit{#1}}}}
\newcommand{\KeywordTok}[1]{\textcolor[rgb]{0.13,0.29,0.53}{\textbf{#1}}}
\newcommand{\NormalTok}[1]{#1}
\newcommand{\OperatorTok}[1]{\textcolor[rgb]{0.81,0.36,0.00}{\textbf{#1}}}
\newcommand{\OtherTok}[1]{\textcolor[rgb]{0.56,0.35,0.01}{#1}}
\newcommand{\PreprocessorTok}[1]{\textcolor[rgb]{0.56,0.35,0.01}{\textit{#1}}}
\newcommand{\RegionMarkerTok}[1]{#1}
\newcommand{\SpecialCharTok}[1]{\textcolor[rgb]{0.00,0.00,0.00}{#1}}
\newcommand{\SpecialStringTok}[1]{\textcolor[rgb]{0.31,0.60,0.02}{#1}}
\newcommand{\StringTok}[1]{\textcolor[rgb]{0.31,0.60,0.02}{#1}}
\newcommand{\VariableTok}[1]{\textcolor[rgb]{0.00,0.00,0.00}{#1}}
\newcommand{\VerbatimStringTok}[1]{\textcolor[rgb]{0.31,0.60,0.02}{#1}}
\newcommand{\WarningTok}[1]{\textcolor[rgb]{0.56,0.35,0.01}{\textbf{\textit{#1}}}}
\usepackage{graphicx}
\makeatletter
\def\maxwidth{\ifdim\Gin@nat@width>\linewidth\linewidth\else\Gin@nat@width\fi}
\def\maxheight{\ifdim\Gin@nat@height>\textheight\textheight\else\Gin@nat@height\fi}
\makeatother
% Scale images if necessary, so that they will not overflow the page
% margins by default, and it is still possible to overwrite the defaults
% using explicit options in \includegraphics[width, height, ...]{}
\setkeys{Gin}{width=\maxwidth,height=\maxheight,keepaspectratio}
% Set default figure placement to htbp
\makeatletter
\def\fps@figure{htbp}
\makeatother
\setlength{\emergencystretch}{3em} % prevent overfull lines
\providecommand{\tightlist}{%
  \setlength{\itemsep}{0pt}\setlength{\parskip}{0pt}}
\setcounter{secnumdepth}{-\maxdimen} % remove section numbering
\ifLuaTeX
  \usepackage{selnolig}  % disable illegal ligatures
\fi
\IfFileExists{bookmark.sty}{\usepackage{bookmark}}{\usepackage{hyperref}}
\IfFileExists{xurl.sty}{\usepackage{xurl}}{} % add URL line breaks if available
\urlstyle{same} % disable monospaced font for URLs
\hypersetup{
  pdftitle={Assignment 2: Coding Basics},
  pdfauthor={Hanna Bliska},
  hidelinks,
  pdfcreator={LaTeX via pandoc}}

\title{Assignment 2: Coding Basics}
\author{Hanna Bliska}
\date{}

\begin{document}
\maketitle

\hypertarget{overview}{%
\subsection{OVERVIEW}\label{overview}}

This exercise accompanies the lessons in Environmental Data Analytics on
coding basics.

\hypertarget{directions}{%
\subsection{Directions}\label{directions}}

\begin{enumerate}
\def\labelenumi{\arabic{enumi}.}
\tightlist
\item
  Rename this file
  \texttt{\textless{}FirstLast\textgreater{}\_A02\_CodingBasics.Rmd}
  (replacing \texttt{\textless{}FirstLast\textgreater{}} with your first
  and last name).
\item
  Change ``Student Name'' on line 3 (above) with your name.
\item
  Work through the steps, \textbf{creating code and output} that fulfill
  each instruction.
\item
  Be sure to \textbf{answer the questions} in this assignment document.
\item
  When you have completed the assignment, \textbf{Knit} the text and
  code into a single PDF file.
\item
  After Knitting, submit the completed exercise (PDF file) to Sakai.
\end{enumerate}

\hypertarget{basics-day-1}{%
\subsection{Basics Day 1}\label{basics-day-1}}

\begin{enumerate}
\def\labelenumi{\arabic{enumi}.}
\item
  Generate a sequence of numbers from one to 100, increasing by fours.
  Assign this sequence a name.
\item
  Compute the mean and median of this sequence.
\item
  Ask R to determine whether the mean is greater than the median.
\item
  Insert comments in your code to describe what you are doing.
\end{enumerate}

\begin{Shaded}
\begin{Highlighting}[]
\CommentTok{\#1. }
\NormalTok{seq\_by\_four }\OtherTok{\textless{}{-}} \FunctionTok{seq}\NormalTok{(}\DecValTok{1}\NormalTok{, }\DecValTok{100}\NormalTok{, }\DecValTok{4}\NormalTok{) }\CommentTok{\#from one, to one hundred, by fours}
\NormalTok{seq\_by\_four}
\end{Highlighting}
\end{Shaded}

\begin{verbatim}
##  [1]  1  5  9 13 17 21 25 29 33 37 41 45 49 53 57 61 65 69 73 77 81 85 89 93 97
\end{verbatim}

\begin{Shaded}
\begin{Highlighting}[]
\CommentTok{\#2. }
\FunctionTok{mean}\NormalTok{(seq\_by\_four) }\CommentTok{\#calculate mean of seq\_by\_four}
\end{Highlighting}
\end{Shaded}

\begin{verbatim}
## [1] 49
\end{verbatim}

\begin{Shaded}
\begin{Highlighting}[]
\FunctionTok{median}\NormalTok{(seq\_by\_four) }\CommentTok{\#calculate median of seq\_by\_four}
\end{Highlighting}
\end{Shaded}

\begin{verbatim}
## [1] 49
\end{verbatim}

\begin{Shaded}
\begin{Highlighting}[]
\CommentTok{\#3. }
\FunctionTok{mean}\NormalTok{(seq\_by\_four) }\SpecialCharTok{\textgreater{}} \FunctionTok{median}\NormalTok{(seq\_by\_four) }\CommentTok{\#asking R if the mean \textgreater{} median}
\end{Highlighting}
\end{Shaded}

\begin{verbatim}
## [1] FALSE
\end{verbatim}

\begin{Shaded}
\begin{Highlighting}[]
\CommentTok{\#output will return TRUE or FALSE}
\end{Highlighting}
\end{Shaded}

\hypertarget{basics-day-2}{%
\subsection{Basics Day 2}\label{basics-day-2}}

\begin{enumerate}
\def\labelenumi{\arabic{enumi}.}
\setcounter{enumi}{4}
\item
  Create a series of vectors, each with four components, consisting of
  (a) names of students, (b) test scores out of a total 100 points, and
  (c) whether or not they have passed the test (TRUE or FALSE) with a
  passing grade of 50.
\item
  Label each vector with a comment on what type of vector it is.
\item
  Combine each of the vectors into a data frame. Assign the data frame
  an informative name.
\item
  Label the columns of your data frame with informative titles.
\end{enumerate}

\begin{Shaded}
\begin{Highlighting}[]
\NormalTok{student\_names }\OtherTok{\textless{}{-}} \FunctionTok{c}\NormalTok{(}\StringTok{"Hanna"}\NormalTok{, }\StringTok{"Caroline"}\NormalTok{, }\StringTok{"Isaac"}\NormalTok{, }\StringTok{"Sam"}\NormalTok{) }
\CommentTok{\#this vector consists of characters}
\NormalTok{test\_scores }\OtherTok{\textless{}{-}} \FunctionTok{c}\NormalTok{(}\DecValTok{48}\NormalTok{, }\DecValTok{90}\NormalTok{, }\DecValTok{38}\NormalTok{, }\DecValTok{95}\NormalTok{) }
\CommentTok{\#this vector consists of numbers}
\NormalTok{passing\_scores }\OtherTok{\textless{}{-}} \FunctionTok{c}\NormalTok{(}\ConstantTok{FALSE}\NormalTok{, }\ConstantTok{TRUE}\NormalTok{, }\ConstantTok{FALSE}\NormalTok{, }\ConstantTok{TRUE}\NormalTok{) }
\CommentTok{\#this vector consists of logical elements}

\NormalTok{df\_student\_names }\OtherTok{\textless{}{-}} \FunctionTok{as.data.frame}\NormalTok{(student\_names) }
\CommentTok{\#start by turning one vector into a data frame}
\NormalTok{df\_student\_scores }\OtherTok{\textless{}{-}} \FunctionTok{cbind}\NormalTok{(df\_student\_names, test\_scores, passing\_scores) }
\CommentTok{\#add columns to the data frame to create a single data frame with all vectors}

\FunctionTok{colnames}\NormalTok{(df\_student\_scores) }\OtherTok{\textless{}{-}} \FunctionTok{c}\NormalTok{(}\StringTok{"student.first.name"}\NormalTok{, }\StringTok{"test.scores.numeric"}\NormalTok{, }\StringTok{"test.scores.passing"}\NormalTok{) }
\CommentTok{\#renaming columns more informative names}
\NormalTok{df\_student\_scores}
\end{Highlighting}
\end{Shaded}

\begin{verbatim}
##   student.first.name test.scores.numeric test.scores.passing
## 1              Hanna                  48               FALSE
## 2           Caroline                  90                TRUE
## 3              Isaac                  38               FALSE
## 4                Sam                  95                TRUE
\end{verbatim}

\begin{enumerate}
\def\labelenumi{\arabic{enumi}.}
\setcounter{enumi}{8}
\tightlist
\item
  QUESTION: How is this data frame different from a matrix?
\end{enumerate}

\begin{quote}
Answer: This data frame includes elements from different classes
(characters, numbers, logical elements). A matrix would include elements
from the same class (e.g., only numbers).
\end{quote}

\begin{enumerate}
\def\labelenumi{\arabic{enumi}.}
\setcounter{enumi}{9}
\item
  Create a function with an if/else statement. Your function should take
  a \textbf{vector} of test scores and print (not return) whether a
  given test score is a passing grade of 50 or above (TRUE or FALSE).
  You will need to choose either the \texttt{if} and \texttt{else}
  statements or the \texttt{ifelse} statement.
\item
  Apply your function to the vector with test scores that you created in
  number 5.
\end{enumerate}

\begin{Shaded}
\begin{Highlighting}[]
\NormalTok{check\_students\_pass }\OtherTok{\textless{}{-}} \ControlFlowTok{function}\NormalTok{(test\_scores) \{}
  \FunctionTok{ifelse}\NormalTok{(test\_scores}\SpecialCharTok{\textgreater{}=}\DecValTok{50}\NormalTok{,}\ConstantTok{TRUE}\NormalTok{,}\ConstantTok{FALSE}\NormalTok{) }\CommentTok{\#log\_exp, if TRUE, if false}
\NormalTok{\}}
\CommentTok{\#check\_students\_pass return TRUE if the test score \textgreater{} 50, FALSE if \textless{} 50}

\NormalTok{did\_my\_students\_pass }\OtherTok{\textless{}{-}}\FunctionTok{check\_students\_pass}\NormalTok{(test\_scores) }
\CommentTok{\#using function check\_students\_pass on my test\_scores}
\NormalTok{did\_my\_students\_pass }\CommentTok{\#output}
\end{Highlighting}
\end{Shaded}

\begin{verbatim}
## [1] FALSE  TRUE FALSE  TRUE
\end{verbatim}

\begin{enumerate}
\def\labelenumi{\arabic{enumi}.}
\setcounter{enumi}{11}
\tightlist
\item
  QUESTION: Which option of \texttt{if} and \texttt{else}
  vs.~\texttt{ifelse} worked? Why?
\end{enumerate}

\begin{quote}
Answer: When I tried to use the \texttt{if} and \texttt{else} option, I
received an error that stated the condition has a length greater than 1.
This is because I was asking \texttt{if} and \texttt{else} to evaluate
all of the test scores in my vector, not a single test score. When I
used \texttt{ifselse} I was able to evaluate all of the test scores in
my vector. This is because \texttt{ifelse} can evaluate vectors.
\end{quote}

\end{document}
